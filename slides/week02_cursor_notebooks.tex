\documentclass[aspectratio=169]{beamer}
\usetheme{Madrid}
\usecolortheme{default}

\title{CSE255: Scalable Data Analysis with Dask}
\subtitle{Week 2: Cursor, Notebooks vs Python Files, Testing, Prompt Engineering}
\author{Course Instructor}
\date{\today}

\begin{document}

\frame{\titlepage}

\begin{frame}
\frametitle{Learning Objectives}
\begin{itemize}
\item Understand productive workflows in Cursor IDE
\item Learn when to use notebooks vs Python modules
\item Set up proper repository structure for data science
\item Implement testing strategies with pytest
\item Master prompt engineering for AI-assisted development
\item Understand reproducibility and guardrails
\end{itemize}
\end{frame}

\begin{frame}
\frametitle{Cursor IDE Workflows}
\begin{block}{Productive Patterns}
\begin{itemize}
\item Use AI for code generation, refactoring, and reviews
\item Pair programming with AI responsibly
\item Leverage context-aware suggestions
\item Use chat for complex questions
\end{itemize}
\end{block}
\begin{block}{Best Practices}
\begin{itemize}
\item Provide clear, specific prompts
\item Review AI-generated code carefully
\item Test everything before committing
\item Use AI to learn, not just generate code
\end{itemize}
\end{block}
\end{frame}

\begin{frame}
\frametitle{Repository Structure}
\begin{block}{Standard Layout}
\begin{verbatim}
project/
├── src/
│   └── package_name/
│       ├── __init__.py
│       ├── preprocessing.py
│       ├── analysis.py
│       └── models.py
├── notebooks/
│   ├── 01_exploration.ipynb
│   ├── 02_analysis.ipynb
│   └── 03_visualization.ipynb
├── tests/
│   ├── test_preprocessing.py
│   └── test_analysis.py
├── data/
├── environment.yml
├── README.md
└── Makefile
\end{verbatim}
\end{block}
\end{frame}

\begin{frame}
\frametitle{Notebooks vs Python Modules}
\begin{columns}
\begin{column}{0.5\textwidth}
\textbf{Use Notebooks For:}
\begin{itemize}
\item Exploratory data analysis
\item Rapid prototyping
\item Visualization and presentation
\item Interactive debugging
\item Learning and experimentation
\end{itemize}
\end{column}
\begin{column}{0.5\textwidth}
\textbf{Use Modules For:}
\begin{itemize}
\item Production code
\item Reusable functions
\item Testable components
\item Data pipelines
\item Complex logic
\end{itemize}
\end{column}
\end{columns}
\begin{block}{Rule of Thumb}
If it needs to run in production or be tested, it belongs in a module
\end{block}
\end{frame}

\begin{frame}
\frametitle{Refactoring Pattern}
\begin{block}{Workflow}
\begin{enumerate}
\item Start in notebook for exploration
\item Identify reusable patterns
\item Extract to module with tests
\item Import module back into notebook
\item Document the decision
\end{enumerate}
\end{block}
\begin{block}{Example}
\begin{verbatim}
# In notebook
def clean_data(df):
    # complex logic
    
# Extract to src/preprocessing.py
# Add tests in tests/test_preprocessing.py
# Import in notebook: from src.preprocessing import clean_data
\end{verbatim}
\end{block}
\end{frame}

\begin{frame}
\frametitle{Testing Strategies}
\begin{block}{pytest Basics}
\begin{itemize}
\item Use pytest for all tests
\item Test files: \texttt{test\_*.py}
\item Test functions: \texttt{def test\_*()}
\item Use fixtures for reusable setup
\end{itemize}
\end{block}
\begin{block}{Testing Types}
\begin{itemize}
\item \textbf{Unit tests}: Individual functions
\item \textbf{Integration tests}: Component interactions
\item \textbf{Data contract tests}: Schema validation
\item \textbf{Lightweight fixtures}: Sample data for testing
\end{itemize}
\end{block}
\end{frame}

\begin{frame}
\frametitle{Data Contract Tests}
\begin{block}{Schema Validation}
\begin{itemize}
\item Verify column names and types
\item Check expected ranges
\item Validate relationships
\item Ensure no nulls in critical columns
\end{itemize}
\end{block}
\begin{block}{Example}
\begin{verbatim}
def test_schema_validation(df):
    assert 'timestamp' in df.columns
    assert df['timestamp'].dtype == 'datetime64[ns]'
    assert df['value'].min() >= 0
    assert df['value'].notna().all()
\end{verbatim}
\end{block}
\end{frame}

\begin{frame}
\frametitle{Prompt Engineering}
\begin{block}{Effective Prompts}
\begin{itemize}
\item Be specific about context
\item Include examples when possible
\item Specify desired output format
\item Ask for explanations
\end{itemize}
\end{block}
\begin{block}{Common Templates}
\begin{itemize}
\item \textbf{Code generation}: "Write a function that..."
\item \textbf{Refactoring}: "Refactor this code to..."
\item \textbf{Review}: "Review this code for..."
\item \textbf{Debugging}: "Why does this code fail when..."
\end{itemize}
\end{block}
\end{frame}

\begin{frame}
\frametitle{Prompt Snippets}
\begin{block}{Code Generation}
\texttt{"Write a function to load Parquet files from S3 using dask, with error handling and logging"}
\end{block}
\begin{block}{Refactoring}
\texttt{"Refactor this pandas code to use dask, maintaining the same API and adding type hints"}
\end{block}
\begin{block}{Code Review}
\texttt{"Review this data preprocessing function for performance, correctness, and best practices"}
\end{block}
\begin{block}{Documentation}
\texttt{"Add docstrings to this function following numpy style, including parameter types and return values"}
\end{block}
\end{frame}

\begin{frame}
\frametitle{Reproducibility}
\begin{block}{Guardrails}
\begin{itemize}
\item Lock dependency versions in \texttt{environment.yml}
\item Use seeded randomness for ML
\item Document all parameters
\item Version control all code
\item Provide clear README
\end{itemize}
\end{block}
\begin{block}{Runnable Pipeline}
\begin{itemize}
\item Single entrypoint: \texttt{make} or \texttt{python -m}
\item Clear documentation of steps
\item Idempotent operations
\item Deterministic results
\end{itemize}
\end{block}
\end{frame}

\begin{frame}
\frametitle{Partition Planning}
\begin{block}{Document Partition Strategy}
\begin{itemize}
\item Partition keys (e.g., date, region)
\item Expected partition sizes
\item Query patterns
\item Trade-offs considered
\end{itemize}
\end{block}
\begin{block}{Example}
\begin{verbatim}
Partitioning Strategy:
- Key: date (YYYY-MM-DD)
- Expected size: ~100MB per partition
- Query pattern: Filter by date range
- Rationale: Most queries are time-based
\end{verbatim}
\end{block}
\end{frame}

\begin{frame}
\frametitle{Week 2 Deliverables}
\begin{block}{Repository Skeleton}
\begin{itemize}
\item Clear module boundaries in \texttt{src/}
\item \texttt{notebooks/} directory with examples
\item \texttt{tests/} directory with pytest setup
\item \texttt{pytest} passing locally
\end{itemize}
\end{block}
\begin{block}{Refactored Example}
\begin{itemize}
\item One notebook refactored into module
\item Minimal documentation on when to use each
\item Tests for the module
\end{itemize}
\end{block}
\begin{block}{Prompt Templates}
Short templates the team will use in Cursor for common tasks
\end{block}
\end{frame}

\begin{frame}
\frametitle{Week 2 Deliverables (cont.)}
\begin{block}{Runnable Pipeline}
\begin{itemize}
\item \texttt{make} or \texttt{python -m} entrypoint
\item Clear README with usage instructions
\end{itemize}
\end{block}
\begin{block}{Partition Plan}
Documented with keys, expected sizes, and rationale
\end{block}
\begin{block}{Cost Visibility}
\begin{itemize}
\item Vacoreum cost trend
\item Storage footprint
\item Table of cost per 1M rows ingested
\end{itemize}
\end{block}
\end{frame}

\begin{frame}
\frametitle{Homework 2 (5\%)}
\begin{block}{CSV/JSON to Parquet Conversion}
\begin{itemize}
\item Convert messy CSV/JSON to Parquet
\item Implement validation checks
\item Document schema
\end{itemize}
\end{block}
\begin{block}{Justify Choices}
\begin{itemize}
\item Partitioning strategy
\item Compression choices
\item Data type decisions
\item Performance considerations
\end{itemize}
\end{block}
\end{frame}

\begin{frame}
\frametitle{Best Practices Summary}
\begin{block}{Code Organization}
\begin{itemize}
\item Clear separation: notebooks for exploration, modules for production
\item Testable, reusable code in modules
\item Document decisions
\end{itemize}
\end{block}
\begin{block}{AI-Assisted Development}
\begin{itemize}
\item Use AI responsibly
\item Review all generated code
\item Learn from AI suggestions
\item Build prompt templates
\end{itemize}
\end{block}
\begin{block}{Testing}
\begin{itemize}
\item Write tests early
\item Use fixtures for data
\item Validate schemas
\item Test edge cases
\end{itemize}
\end{block}
\end{frame}

\begin{frame}
\frametitle{Next Steps}
\begin{enumerate}
\item Set up repository structure
\item Refactor one notebook example
\item Create prompt templates
\item Write tests for modules
\item Document partition plan
\item Complete HW2
\end{enumerate}
\begin{block}{Questions?}
\end{block}
\end{frame}

\end{document}

